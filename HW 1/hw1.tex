\documentclass[11pt,twoside,a4paper]{article}
\usepackage{mathtools}
\usepackage{forest}
\DeclarePairedDelimiter\ceil{\lceil}{\rceil}
\DeclarePairedDelimiter\floor{\lfloor}{\rfloor}
\begin{document}
\author{Ankur Dhoot}

CS 381 HW 1

Ankur Dhoot \newline

1. 

a) T(n) = 9T(n/3) + $n^{2}$ for $n > 2$ and T(n) = 1 otherwise

	a = 9 \qquad b = 3 \qquad	f(n) = $n^{2}$
	
Since $n^{\log_b a}$ = $n^{\log_3 9}$ = $n^{2}$ and f(n) = $\Theta(n^{2})$, f(n) = $\Theta(n^{\log_b a})$. Thus, we are in case 2 of the master theorem. Therefore, T(n) = $\Theta(n^{2} \lg n)$. \newline

b) T(n) = 6T(n/2) + $n^{2.4}$ for $n > 1$ and T(n) = 1 otherwise
 
 a = 6 \qquad b = 3 \qquad f(n) = $n^{2.4}$
 
 Since $n^{\log_b a}$ = $n^{\log_2 6}$ = $n^{2.58}$ and f(n) = $\Theta(n^{2.4})$, f(n) = $O(n^{2.58 - \epsilon})$ for $0 < \epsilon < .18$. Thus, we are in case 1 of the master theorem. Therefore, T(n) = $\Theta(n^{\log_2 6}) = \Theta(n^{2.58})$. \newline
 
c) T(n) = 12T(n/4) + $n^{2}$ for $n > 3$ and T(n) = 1 otherwise

a = 12 \qquad b = 4 \qquad f(n) = $n^{2}$

Since $n^{\log_b a}$ = $n^{\log_4 12}$ = $n^{1.79}$ and f(n) = $\Theta(n^{2})$, f(n) = $\Omega(n^{1.79 + \epsilon})$ for $0 < \epsilon < .20$. Thus, we are in case 3 of the master theorem. Therefore, T(n) = $\Theta(f(n)) = \Theta(n^{2})$. \newline

2. 
Proof: By induction on n

Base case: n = \{1,2,3\}

For all n $\in$ \{1,2,3\}, T(n) = 1 $\leq$ 12n.
Thus, the statement holds for the base case.

Induction Step:

\qquad Inductive Hypothesis: Suppose that the statement holds for all positive $m < n$. That is T(m) $\leq$ 12m. 

Now, we must show that the statement holds for n. That is, we must show that T(n) $\leq$ 12n.

T(n) = T($\floor*{2n/3}$) + T($\floor*{n/4}$) + n

$\leq 12\floor*{2n/3} + 12\floor*{n/4} + n$ (by induction hypothesis)

$\leq 12(2n/3) + 12(n/4) + n$ (by definition of floor function)

$=8n + 3n + n$

=12n.

Thus, having established the basis and induction step, the claim follows by the principle of mathematical induction. \newline

3.
T(n) = 1 for $n \leq 2$

T(n) = T($\floor*{3n/5}$) + T($\floor*{2n/5}$) + $cn$ for $n > 2$

Recursion Tree argument:

The longest simple path from root to a leaf is $n \rightarrow \floor{3n/5} \rightarrow \floor{3\floor{3n/5}/5} \rightarrow ... \rightarrow 1$ Thus, the height of the tree is at most $\log_{5/3} n$. Consider the first level of the tree. The size of the "subproblem" is $\floor*{3n/5}$ and $\floor*{2n/5}$ which will add cost c$\floor*{3n/5}$ + c$\floor*{2n/5}\leq cn$. The second level will have 4 "subproblems" of size $\floor{3\floor{3n/5}/5}, \floor{2\floor{3n/5}/5}, \floor{3\floor{2n/5}/5}, \floor{2\floor{2n/5}/5}$, which will contribute cost c$\floor{3\floor{3n/5}/5} + c\floor{2\floor{3n/5}/5} + c\floor{3\floor{2n/5}/5} + c\floor{2\floor{2n/5}/5}$ $\leq c(9n/25) + c(6n/25) + c(6n/25) + c(4n/25) = cn.$ It's easy to see that every level of the tree will contribute at most $cn$ to the cost of the tree with the cost per level decreasing as we move further down the tree since the tree isn't a complete binary tree and more internal nodes will be missing. Since the total cost is the sum of the cost at each level which is bounded by $cn$ and there are at most $\log_{5/3} n$ levels, the total cost $T(n)$ is $O(cn * \log_{5/3} n) = O(n\lg n)$.


Sketch of first two levels of recursion tree:\newline


\begin{forest}
  for tree={
    draw,
    align=center
  }
  [$n$
    [$\floor*{2n/5}$
      [$\floor*{2\floor{2n/5}/5}$]
      [$\floor*{3\floor{2n/5}/5}$]
    ]
    [$\floor*{3n/5}$
      [$\floor{2\floor{3n/5}/5}$]
      [$\floor{3\floor{3n/5}/5}$]
    ]
  ]
\end{forest}

\end{document}
