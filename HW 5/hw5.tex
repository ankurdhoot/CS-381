\documentclass[11pt,a4paper]{article}
\usepackage{algorithm}
\usepackage{algpseudocode}
\usepackage{amsmath}
\begin{document}
\author{Ankur Dhoot}
\title{CS 381 HW 5}
\maketitle

\section*{Q1}
After running DFS on the graph of figure 22.6 :

dfs(q)

\quad dfs(w)

\quad \quad dfs(s)

\quad \quad \quad dfs(v)

\quad dfs(t)

\quad \quad dfs(y)

\quad \quad dfs(x)

\quad \quad \quad dfs(z)

dfs(r)

\quad dfs(u)
\newline

dfs(q) discovers (in order) w, s, v, t, y, x, z

dfs(r) discovers u

Thus, the start and finish times are :

q(1, 16)

r(17, 20)

s(3, 6)

t(8, 15)

u(18, 19)

v(4, 5)

w(2, 7)

x(11, 14)

y(9, 10)

z(12, 13)

After reversing G and running DFS on the vertices in order of decreasing finishing time:

dfs(r) discovers r

dfs(u) discovers u

dfs(q) discovers q, y, t

dfs(x) discovers x, z

dfs(w) discovers w, v, s

Thus, the SCCs produced are \{r\}, \{u\}, \{q, y, t\}, \{x, z\}, \{w, v, s\}.

\section*{Q2}

Let $N_{ij}^{k}$ denote the number of different shortest paths from i to j for which all intermediate vertices are in the set \{1,2,...k\}. Using the same convention for $w_{ij}$ as the book, we get

$N_{ij}^{0}$ =  
\[ \begin{cases} 
      1 & w_{ij} < \infty \\
      0 & w_{ij} = \infty \\
   \end{cases}
\]

since the number of shortest paths between i and j using no intermediate vertices is 1 if there exists an edge between i and j and 0 otherwise.

For k $>$ 0, we can split $N_{ij}^{k}$ into three cases:

$N_{ij}^{k}$ = 

\[ \begin{cases} 
      N_{ij}^{k-1} & d_{ij}^{k-1} < d_{ik}^{k-1} + d_{kj}^{k-1} \\
      N_{ij}^{k-1} + \delta_{ij}^{k} & d_{ij}^{k-1} = d_{ik}^{k-1} + d_{kj}^{k-1} \\
      \delta_{ij}^{k} & d_{ij}^{k-1} > d_{ik}^{k-1} + d_{kj}^{k-1}
   \end{cases}
\]

where $\delta_{ij}^{k}$ = $N_{ik}^{k-1}$ x $N_{kj}^{k-1}$

The reasoning is as follows:

1. If the shortest path between i and j using intermediate vertices \{1,2,...k\} doesn't use vertex k (i.e $d_{ij}^{k-1} < d_{ij}^{k-1} + d_{kj}^{k-1}$), the the number of shortest paths between i and j using intermediate vertices \{1,2,...k\} is just the number of shortest paths using intermediate vertices \{1,2,...k-1\}. 

2. If $d_{ij}^{k-1} = d_{ik}^{k-1} + d_{kj}^{k-1}$, then the number of shortest paths from i to j using intermediate vertices \{1,2,...k\} is the number of shortest paths using intermediate vertices \{1,2,...k-1\} plus the number of shortest paths going through intermediate vertex k. The number of shortest paths going through vertex k is $\delta_{ij}^{k}$ = $N_{ik}^{k-1}$ x $N_{kj}^{k-1}$ (number of shortest paths from i to k multiplied by number of shortest paths from k to j) Obviously, k cannot be an intermediate vertex on a path from i to k or k to j (else we'd have a negative cycle contrary to our hypothesis) so we can use the set of intermediate vertices \{1,2,...k-1\} to calculate $\delta_{ij}^{k}$.

3. If $d_{ij}^{k-1} > d_{ik}^{k-1} + d_{kj}^{k-1}$, then the number of shortest paths from i to j is just the number of shortest paths that go through k which is $\delta_{ij}^{k}$.

Note that we can drop the superscripts. Why? If having dropped the superscripts, we were to compute and store $N_{ik}$ or $N_{jk}$ before using these values to compute $\delta_{ij}$, then we might have one of the following situations:

$\delta_{ij}^{k}$ = 

\[ \begin{cases} 
      N_{ik}^{k}$ x $N_{kj}^{k-1} &  \\
      N_{ik}^{k-1}$ x $N_{kj}^{k} &  \\
      N_{ik}^{k}$ x $N_{kj}^{k} & 
   \end{cases}
\]

However, k cannot be an intermediate vertex on any shortest path from i to k (else negative cycle contrary to hypothesis), so the number of shortest paths from i to k using intermediate vertices \{1,2,...k\} is just the number of shortest paths from i to k using vertices \{1,2,...k-1\}. Thus, $N_{ik}^{k}$ = $N_{ik}^{k-1}$. Similarly, $N_{kj}^{k}$ = $N_{kj}^{k-1}$. Thus, we can drop the superscripts.


\begin{algorithm}
\caption{Floyd-Warshall with total number of shortest paths}
\begin{algorithmic}[1]
\Function{Floyd-Warshall}{W}
 \State n = W.rows
 \State D = W
 \State Initialize N ($n_{ij} = N_{ij}^{0}$ as described above)
 \For{k = 1 to n}
 	\For{i = 1 to n}
 		\For{j = 1 to n}
 			\If{$d_{ij}$ $<$ $d_{ik}$ + $d_{kj}$}
 				\State $n_{ij} = n_{ij}$
 			\ElsIf{$d_{ij}$ $=$ $d_{ik}$ + $d_{kj}$}
 				\State $n_{ij}$ = $n_{ij}$ + $n_{ik}$ x $n_{kj}$
 			\Else
 				\State $n_{ij}$ = $n_{ik}$ x $n_{kj}$
 			\EndIf
 			\State $d_{ij}$ = min($d_{ij}$, $d_{ik}$ + $d_{kj}$)
 		\EndFor
 	\EndFor
 \EndFor
 \State \Return D, N
\EndFunction
\end{algorithmic}
\end{algorithm}

The runtime is clearly O($V^{3}$) due to the three for loops and all computation inside the innermost for loop is O(1) depending only on previously computed values. The space required is O($V^2$) since we dropped the superscripts and now only store the V x V matrices N and D, compared to the common implementation which uses O($V^3$) space to store the matrices computed at each iteration.



\end{document}