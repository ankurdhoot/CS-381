\documentclass[11pt,a4paper]{article}
\usepackage{algorithm}
\usepackage{algpseudocode}
\usepackage{amsmath}
\begin{document}
\author{Ankur Dhoot}
\title{CS 381 HW 7}
\maketitle

\section*{Q1}

	\paragraph{(a)}
	Finding a min-cut in G is simple. Suppose we run Edmonds-Karp on G. Look at the final residual graph, $G_{f}$, which doesn't contain any augmenting paths. Let S be the set of vertices reachable from s and let T = V - S. Clearly s $\in$ S and t $\in$ T and S $\cup$ T = V meaning (S,T) is a valid cut. As is shown in Theorem 26.6 in CLRS, the cut is a min-cut since the value of the flow equals the capacity of the cut. We can efficiently find the set S by running BFS or DFS in $G_{f}$ from s and then the min-cut is simple (S, V-S). The time taken to run Edmonds-Karp is O($VE^{2}$) and the search to find the set S runs in time O(V + E). Thus, the run-time is dominated by Edmonds-Karp and is O($VE^{2}$).
	
	\begin{algorithm}
	\caption{Find a min-cut in G}
	\begin{algorithmic}[1]
	\Function{findMinCut}{G, s, t}
	\State Run Ford-Fulkerson on G
	\State BFS in $G_{f}$ from s to get (S, V - S)
	\State \Return (S, V-S)
	\EndFunction
	\end{algorithmic}
	\end{algorithm}
	
	\paragraph{(b)}
	I will prove the following claim which will result in an immediate algorithm for determining whether G has a unique min-cut. \newline

	Claim: Run Ford-Fulkerson on G and let $G_{f}$ be the resulting residual graph. Let S be the set of vertices reachable from s in $G_{f}$ and let T be the set of vertices that have a path to t in $G_{f}$. (Note that S $\cup$ T = V is not necessarily true, which is essentially the crux of the proof). Then (S, V-S) $\neq$ (V-T, T) if and only if there exists more than one min-cut in G. \newline
	
	Proof: $\Longrightarrow$ Suppose (S, V-S) $\neq$ (V-T, T). We know that (S, V-S) defines a min-cut. It's also clear that (V-T, T) defines a min-cut (Exact same proof as given in Theorem 26.6 in CLRS easily shows this). Thus, we have more than one min-cut in G. 
	
	$\Longleftarrow$ Suppose there exists more than one min-cut in G. In $G_{f}$, we know there exists no residual edges from S to V-S, and we know (S, V-S) is a min-cut. Let (X,Y) be another min-cut of G. No vertices of Y can be in S. Why? A min-cut partitions the vertices such that there are no residual edges from X to Y in $G_{f}$ meaning there exists no path from s to any vertex in Y in $G_{f}$. Thus, by the definition of S, no vertex in Y is a vertex in S. $\Rightarrow$ Y $\subset$ V - S. We know Y $\neq$ V - S since (X,Y) $\neq$ (S, V-S). We also know that T $\subset$ V - S, since no vertex in T can be in S, else we'd have an augmenting path in $G_{f}$. Now we need to show that T $\neq$ V - S. Suppose, for the sake of contradiction that T = V - S. $\Rightarrow$ all vertices in V - S have a path to t. But Y $\subset$ V - S and Y $\neq$ V - S $\Rightarrow$ (X,Y) splits V - S such that not all vertices in V - S can have a path to t. Why? Suppose all vertices in V - S have a path to t $\Rightarrow$ some vertex x $\in$ X has a path to t. At some point, the path from x to t must go from a vertex in X to a vertex in Y (since t $\in$ Y). But this contradicts the fact that (X, Y) is a min-cut meaning there are no residual edges from X to Y in $G_{f}$. Thus, not all vertices in V - S have a path to t $\Rightarrow$ T $\neq$ V - S.
	
	Having proved both directions of the statement, the proof is complete.
	
	The procedure follows immediately : 
	\begin{algorithm}
	\caption{Find and print two distinct min-cuts if they exist}
	\begin{algorithmic}[1]
	\Function{printDistintMinCuts}{G, s, t}
	\State Run Ford-Fulkerson on G
	\State BFS in $G_{f}$ from s to get (S, V - S)
	\State Reverse $G_{f}$ to get $G_{f}^{R}$
	\State BFS in $G_{f}^{R}$ from t to get (V - T, T)
	\If {(S,V-S) == (V-T, T)}
		\State print("Min-cut is unique")
	\Else 
		\State print(S, V-S)
		\State print(V-T, T)
	\EndIf
	\EndFunction
	\end{algorithmic}
	\end{algorithm}
	
	The run-time of Ford-Fulkerson using Edmonds-Karp is O($VE^{2}$). Lines 3 and 5 runs in O(V+E) and line 4 can be done in O(V+E) as well. Thus, the run-time is dominated by Edmonds-Karp and is O($VE^{2}$).
	

\newpage
\section*{Q2}

We'll reduce the problem to finding the median in a set of numbers which we know can be done in linear time. 

Given the line y = mx + b, let $d_{i}$ = (m$x_{i}$ + b) - $y_{i}$, which represents the (signed) vertical distance between the ith point and the line. Find the median of these $d_{i}$ values, and call it $d^{*}$ and its associated point ($x^{*}, y^{*}$). Let the new line that is parallel to the original line, but divides the points into equal-sized subsets pass through ($x^{*}, y^{*}$). Using middle-school rules for lines, we get that the new line has equation y = mx + ($y^{*} - mx^{*}$). Since the new line is simply a vertical shift of the original, it's clear that the relative ordering of the $d_{i}$ values will remain the same when computed using the new line. Thus, the median, $d^{*}$ and its associated point ($x^{*}, y^{*}$), remain the same $\Rightarrow$ half the points have d values $\geq$ $d^{*}$ and half the points have d values $\leq$  $d^{*}$. The line passes through ($x^{*}, y^{*}$) $\Rightarrow$ half the points lie on or above the line and half the points lie on or below the line.

The algorithm follows :

	\begin{algorithm}
	\caption{Find parallel line that divides points into equal-sized subsets}
	\begin{algorithmic}[1]
	\Function{findLine}{P}
	\State compute the d values for all points in P
	\State Let $d^{*}$ be the median of the d values with corresponding point ($x^{*}, y^{*}$) 
	\State Compute the parallel line through ($x^{*}, y^{*}$)
	\State \Return this new parallel line 
	\EndFunction
	\end{algorithmic}
	\end{algorithm}
	
	Line 2 takes O(n) time. Line 3 can be done in O(n) time. Line 4 takes constant time. Thus, the algorithm runs in O(n) time.


\end{document}